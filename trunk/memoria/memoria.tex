\documentclass{book}

\usepackage[utf8]{inputenc}
\usepackage[T1]{fontenc}
\usepackage[catalan]{babel}
\usepackage{url}
\usepackage{hyperref}
\usepackage[numbers]{natbib}
\usepackage{wrapfig}
\usepackage{subfig}
\usepackage{sidecap}
\usepackage{verbatim}
\usepackage{setspace}

\usepackage[margin=2.5cm]{geometry}

\usepackage[pdftex]{graphicx}

\begin{document}

\pagestyle{empty}

\begin{titlepage}

\begin{center}


% Logo UAB
\includegraphics[width=1.00\textwidth]{./img/logoUAB.png}\\[1cm]    

% Title
\textsc{\LARGE Quadriga: Plataforma de Programació en Sistemes d'Entitats, per al desenvolupament de Videojocs.}\\[1.5cm]



\end{center}

\begin{flushright}

\vfill

% Dades
\begin{minipage}{0.4\textwidth}
\end{minipage}
\begin{minipage}{0.6\textwidth}
Memòria del Projecte Fi de Carrera \\
d'Enginyeria en Informàtica \\
realitzat per \\
{\bf Isaac Serrano Guasch} \\
i dirigit per \\
{\bf Enric Martí Godia} \\
Bellaterra, \today
\end{minipage}

\end{flushright}

\end{titlepage}
\newpage
\input{./certificatDireccio}
\newpage
\newpage

{\LARGE Agraïments}
\\

Primer, a en Jordi Arnal per a guiar-me a sistemes semblants al que volia desenvolupar, però ja acabats, i no tallar-se un pèl a l'hora de senyalar defectes al meu.
\\

Als meus companys del Màster de Creació de Videojocs: en Sergi, n'Hèctor, en Carles, n'Eduard, n'Albert i en José Manuel, per ajudar-me a viure un projecte més real i fer-me veure quines són les seves necessitats, i als professors del mateix Màster per donar-me les eines per resoldre-les.
\\

I finalment, a la meva parella, n'Ada, per repassar aquesta memòria quan calia, la meva família i als meus amics per acompanyar-me i animar-me en el trajecte de la carrera del qual aquest projecte n'és només l'estació final.

\pagestyle{headings}
\setcounter{page}{1}
\pagenumbering{roman}

\tableofcontents
\listoffigures
\listoftables


\chapter{Introducció}

Des de l'aparició dels primers videojocs, la seva evolució ha estat sempre emparellada amb una evolució tecnològica constant i, en certs punts, accelerada. Aquesta evolució tecnològica no només consta de millores en el rendiment i qualitat de cada joc sinó que, a més, s'ha aconseguit generalitzar solucions i aplicar tecnologia d'un videojoc a altres i fins hi tot crear eines que permeten el desenvolupament de jocs a partir d'elles.

Si mirem aquesta evolució des de més a prop, veiem com al principi els videojocs eren evolucions d'altres jocs típics de bar, màquines escurabutxaques com el pinball serviren d'inspiració per a crear els primers videojocs comercials, amb un model de negoci molt semblant. Posteriorment, la necessitat de crear grans quantitats de jocs va obligar als desenvolupadors a crear màquines que poguessin executar més d'un joc i així abaratir costos i finalment, amb l'arribada dels ordinadors personals i les consoles de sobretaula, es comença a desenvolupar un mercat per a jocs sobre plataformes genèriques.
\\

Poc a poc van apareixent els motors gràfics - programes o mòduls encarregats del renderitzat d'un joc o d'un programa amb gràfics 2D o 3D - o fins hi tot motors de joc -una plataforma per desenvolupar-hi un joc a sobre -. En un principi aquests motors s'utilitzaven dintre de la mateixa companyia que el creava. Per exemple LucarArts creà {SCUMM} ({Script Creation Utility for Manic Mansion}) a l'hora que creava la seva aventura gràfica de "Point \& Click" Manic Mansion. Aquest mateix programa fou utilitzat després en altres jocs com Indiana Jones i l'Última Creuada, LOOM, El Dia del Tentacle i tres jocs de la saga Monkey Island.

Un pas més endavant el va dur Id Software amb el seu id Tech. Aquest motor - i les seves evolucions - no només es feu servir per fer jocs com Doom i Doom II d'Id Software, sinó que es va vendre a altres companyies per a fer altres jocs, tot i que aquests jocs serien molt semblants al Doom original. Posteriorment fins hi tot hi va haver companyies que basaven el seu negoci no en vendre jocs, sinó en vendre motors a altres companyies que els fessin servir; és doncs l'aparició definitiva dels motors com a Middleware.
\\

\section{Els sistemes d'entitats}

Paral·lelament a l'evolució tecnològica ja esmentada, la metodologia de desenvolupament i els mateixos llenguatges de programació han anat evolucionant. L'evolució més important fou quan es va passar de programar bàsicament en {\bf C} a {\bf C++}. El canvi de paradigma, però, s'ha demostrat difícil i, tot i que l'ús de la metodologia orientada a objectes és predominant a quasi totes les àrees d'un motor, encara n'hi ha alguna on porta problemes.
\\

El cas més important, i el que en aquest treball ens centrarem és en la definició de la lògica d'un joc.

\subsection{L'aproximació clàssica}

Quan un programador de {C++} o qualsevol llenguatge orientat a objectes s'asseu davant el desafiament de programar la lògica d'un joc, trenca els diferents objectes que el poblen i els distribueix en diferents grups i subgrups, després en programa les funcionalitats comunes i acaba creant una jerarquia de classes que defineixen tots els objectes i les seves interrelacions. 

Aquesta aproximació sembla senzilla, però acaba comportant diversos problemes. Com s'explica a \citep[p.~719]{Gregory09}, les jerarquies massa grans d'objectes pateixen dels següents inconvenients:

\begin{description}
  \item[Manteniment] \hfill \\
    A més profunda és una classe dintre d'una jerarquia, més costa d'entendre, mantenir i modificar; ja que s'ha d'entendre tant ella com totes les seves classes superiors. Així com modificar una classe mare pot comportar problemes  a classes derivades molt difícils de detectar i arreglar.
    
  \item[Impossibilitat de descriure taxonomies multi-dimensionals] \hfill \\
    Crear classes en forma d'arbre és molt pràctic i sobretot intuïtiu, especialment on a cada nivell es fan separacions respecte un criteri per cada nivell. El problema seriós arriba quan ens trobem amb classificacions que no havíem previst inicialment. Per exemple podríem haver classificat 2 tipus de vehicles: vehicles terrestres i marítims, per posteriorment haver d'afegir vehicles amfibis, repte que ens porta als 2 següents punts.
    
  \item[Herència múltiple. El diamant de la mort] \hfill \\
    La solució naïf del problema anterior seria crear la classe {\em VehicleAmfibi} hereva tant de {\em VehicleTerrestre} com de {\em VehicleMarítim}, cosa que ens porta directament a l'herència múltiple. Com s'explica a \cite[p.~2]{Martin97}, l'herència múltiple causa diversos problemes, moltes vegades més grans que aquells que soluciona. En aquest cas veuríem que la classe {\em VehicleAmfibi} heretaria dues vegades {\em Vehicle}, amb els problemes d'ambigüitat que això duria.
    
  \item[Classes Mix-in] \hfill \\
    Una altra solució és crear un seguit de classes que aportin funcionalitat a diversos llocs de la jerarquia. Aquestes classes, per funcionar bé, cal que siguin heretades només per les fulles i que cap d'elles tingui una classe mare. Per tant diríem que una classe només pot tenir un "avi" en qualsevol cas. Aquesta solució comporta, sobretot, molta disciplina i acaba resultant una solució molt semblant a "agregar" funcionalitat en comptes de heretar-la.
    
  \item[Efecte bombolla] \hfill \\
    A l'inici del disseny, les classes arrels - les més pròximes a l'inici de la jerarquia - són dissenyades inicialment amb poca funcionalitat. A mesura que avança el projecte, i davant del desig de compartir codi i, sobretot, no duplicar-lo, molta funcionalitat va pujant a la jerarquia fins que troba el "comú denominador". A poc a poc, les classes arrels es van fent pesades fins que contenen la major part de funcionalitat, que les seves filles s'han d'encarregar d'activar correctament. Col·lateralment això fa que moltes classes acabin tenint funcionalitat i variables que realment no necessiten, fent que el programa usi més memòria de la necessària, un problema especialment greu en jocs de consola, on la memòria és un bé molt escàs.
    
\end{description}

\subsection{De l'{\em és-un} al {\em conté-un}}

La primera millora, o petit canvi, que es proposa respecte l'aproximació anterior és agregar la funcionalitat en comptes de heretar-la. Si volem crear un objecte amb moviment, que es renderitzi, que col·lisioni i que s'animi - un enemic, per exemple -, abans heretaríem d'una jerarquia on, més amunt o més avall, estigui implementada tota aquesta funcionalitat. Ara, però, el que hauríem de fer és simplement crear un seguit de propietats a la classe {\em Enemic} que apuntin a instàncies d'altres classes que ens aportin cada una la funcionalitat que busquem.

Dintre d'aquest esquema, les classes que aporten funcionalitat són moltes vegades anomenades {\em Components} o {\em objectes-servidors}, ja que són les que componen els objectes finals.
\\

Aquesta aproximació millora en certs aspectes l'anterior, però encara té alguns inconvenients. Es tendeix a tenir una classe {\em GameObject} amb un punter a cada tipus de component, al qual se li afegeixen o treuen els components segons convingui - un {\em Enemic} contindrà el component {\em ObjecteAnimat} mentre que una {\em Habitació} no -; cosa que comporta primer el problema de qui s'ha d'encarregar de destruir els components - evident fins que arribes al punt que vols substituir un component en temps d'execució - i després la pèrdua d'eficiència que comporta haver de comprovar constantment si una instància conté o no cada component per fer-ne operacions.
\\

Un pas més enllà es troba l'aproximació on cada tipus de {\em Component} implementa una interfície comuna molt bàsica i després el {\em GameObject} simplement conté un array dinàmic que els conté tots, i crida les seves funcions ordenadament. Aquesta aproximació comporta sovint el problema de que els components cal ordenar-los, ja que moltes vegades la seva funcionalitat no és commutativa.
\\

Un últim pas en aquesta línia és eliminar per complet el {\em GameObject} i tenir arrays de components i lligar-los simplement per l'índex - Arrays o una implementació menys costosa en memòria com Taules Hash -. Aquesta aproximació sovint anomenada com a {\em Model de Components Pur} guanya sobretot en coherència de caché. Normalment tots els components d'un tipus s'actualitzen alhora, al estar tots en un array seguits, els {\em misses} de caché són mínims.
\\

Finalment, un gran avantatge d'un sistema d'aquest tipus és, com s'explica a \cite{Leonard99}, la possibilitat de, un cop definits els tipus de components i les seves propietats, crear un "editor del joc" que permeti, amb l'ajuda d'una interfície gràfica, crear, manipular, modificar, afegir, treure, etc... objectes i components del món de forma molt senzilla. Una eina d'aquestes característiques disminueix la dependència que hi ha entre dissenyadors i programadors i permet que cada un d'ells pugui fer la seva feina en paral·lel de la forma més eficient possible.

\subsection{De centrar-nos en els objectes, a centrar-nos en les propietats}

L'últim pas, que abstracta totes les idees anteriorment exposades, és trencar totalment l'esquema orientat a objectes. La clau aquí és oblidar que un objecte està format per un seguit de propietats i mètodes, i quedar-se només amb les propietats. D'aquesta manera podem crear una taula per cada tipus de component que un objecte del joc pot tenir, i fer-ne una columna per cada propietat i una fila per cada objecte que el contingui, tenint cada objecte del joc, el que abans coneixíem com un {\em GameObject}, un identificador únic.

Aquesta aproximació no està pas mancada de defectes, i cal tenir-los en compte. El primer d'ells és la dificultat que hi ha d'establir relacions entre les {\em entitats} del joc, ja que no existeix una implementació en sí, d'aquesta entitat. Relacionat amb això hi ha el problema d'inicialitzar aquestes {\em entitats}, que s'ha de fer d'una forma més o menys aliena al mateix sistema. Finalment, el dilema més gran, és decidir on es programa el comportament de les entitats, i hi ha bàsicament 2 solucions, amb les seves variants:

\begin{description}
  \item[Dintre els mateixos components] \hfill \\
    Cada propietat porta incorporada - possiblement - la seva funcionalitat d'alguna manera. Sigui directament dintre del mateix components, via mètodes en aquest - cosa que retorna a la metodologia orientada a objectes - o via els anomenats {\em sistemes}. Aquests s'encarreguen de recollir aquelles {\em entitats} que reuneixen unes condicions - normalment, tenir un o més components - i actualitzar-los.
    
  \item[Via components especials] \hfill \\
    En aquesta variant normalment es defineix un tipus especials que simplement conté un apuntador - o referència equivalent - a un script que implementi una interfície coneguda i que defineixi el comportament d'aquesta Entitat.
    
\end{description}



\section{Implementacions Prèvies}

\subsection{Jocs Concrets}

\subsection{Motors de jocs}

\subsection{El nostre tret diferencial}

\section{Estructura de la memòria}

\chapter{Desenvolupament del Projecte}
\label{chap:Desenvolupament}

En aquest capítol es parlarà primer del disseny i implementació del llenguatge en trets generals i s'entrarà en algun detall de la seva implementació. Seguidament es farà una introducció més detallada de les construccions pròpies del llenguatge i la seva funcionalitat. A continuació s'analitzarà el model de dades d'un programa compilat amb Quadriga i finalment es farà una petita introducció a les eines que s'han desenvolupat per tal d'ajudar la programació en Quadriga.

\section{Disseny i implementació}

  A l'hora de fer el disseny del projecte, primer s'ha hagut d'esquematitzar el funcionament final d'aquest. El funcionament proposat es detalla a continuació i es mostra a la figura \ref{fig:EsquemaExecucio}.
  
  \begin{figure}
    \includegraphics[width=1\linewidth]{./img/EsquemaExecucio.png}
    \caption{Esquema de l'execució \label{fig:EsquemaExecucio}}
  \end{figure}
  
  \begin{enumerate}
    \item Es crida l'execució de Quadriga passant-li per paràmetre l'objecte de tipus Main que volem executar, i en quin fitxer es troba.
    \item El programa compila aquest fitxer, cerca quines dependències (elements definits en altres fitxers) té i compila aquestes dependències. En cap cas cal especificar explícitament directives {\bf \#{}include} similars a {\bf C}, sinó que amb el nom de l'element el compilador n'ha de tenir prou per saber on buscar-lo.
    \item Un cop compilat el programa, es crea un model de dades especial per a aquest, que depèn dels components definits, ja que per cada un d'aquests s'han de guardar unes dades diferents.
    \item Tenint el model de dades i un codi executable, l'intèrpret carrega les llibreries que necessita i executa aquest codi, que treballa sobre el model de dades creat.
  \end{enumerate}

  Aquest model teòric s'ha vist finalment implementat en un seguit de mòduls, tal com es mostren a la figura  \ref{fig:DiagramaDeModuls}.

  \begin{figure}
    \includegraphics[width=1\linewidth]{./img/Moduls.png}
    \caption{Diagrama de mòduls \label{fig:DiagramaDeModuls}}
  \end{figure}
  
  El primer mòdul és el Compilador. Aquest està creat amb l'ajuda de JavaCC, una llibreria especialitzada en crear {\bf Analitzadors sintàctics LL} citep{WikiAnalitzadorLL}. S'ha modificat un analitzador de la gramàtica de Java per tal de simplificar-ne la creació. En conseqüència el llenguatge creat és molt similar a Java, sobretot en el cos de funcions, ja que els canvis que s'hi han fet es resumeixen en:
  
  \begin{itemize}
    \item Eliminar la declaració de classes.
    \item Afegir la declaració d'elements propis de Quadriga: {\em Component, Event, System, Prototype, Thread, Main}
    \item Afegir el tipus de dades {\em Entity}
    \item Afegir instruccions per a manipular entitats.
  \end{itemize}
  
  Aquest analitzador sintàctic crea una taula amb totes les declaracions, i aquestes contenen en forma d'arbre el codi executable. Aquest codi és interpretat de forma recursiva pel mateix arbre. Per iniciar l'execució es cerca l'objecte de tipus \"{}Main\"{} que l'usuari ha especificat per paràmetre.
  
  Un segon mòdul és l'entorn d'execució. Aquest s'encarrega de proporcionar a l'intèrpret les dades que necessita, de gestionar els diferents events que es puguin succeir, de crear i eliminar entitats o afegir i treure els components d'aquestes. Aquest entorn s'ajuda d'una base de dades per a gestionar la majoria de la informació.
  
  Finalment s'ha optat per crear també un seguit de llibreries addicionals, que facin de mòduls complementaris d'un {\em Engine}, el que seria una petita llibreria estàndard de Quadriga, amb l'objectiu principal de poder fer-ne una bona demostració.
  
  %%%%%%%%%%%%%%%%%%%%%%%%%%%%%%%%%%%%%%%%%%%%%%%%%%%%%%%%
  %%%%%%%%%%%%%%%%%%%%%%%%%%%%%%%%%%%%%%%%%%%%%%%%%%%%%%%%
  %%%%%%%%%%%%%%%%%%%%%%%%%%%%%%%%%%%%%%%%%%%%%%%%%%%%%%%%
  %%%%%%%%%%%%%%%%%%%%%%%%%%%%%%%%%%%%%%%%%%%%%%%%%%%%%%%%
  %%%%%%%%%%%%%%%%%%%%%%%%%%%%%%%%%%%%%%%%%%%%%%%%%%%%%%%%
  
  Com a últim apunt, fem un resum de com es pensen aconseguir els objectius bàsics de disseny:

  \begin{itemize}
    \item {\bf Open Source:} El codi del projecte s'ha penjat públicament a GoogleCode sota una llicència LGPL, que permet usar-la lliurement amb l'única restricció d'haver de publicar els canvis fets sobre la mateixa llibreria sota la mateixa llicència. També s'ha optat per fer servir només llibreries OpenSource com l'analitzador de sintaxi ({\em JavaCC}), la llibreria gràfica ({\em LWJGL}) i la base de dades ({\em HSQLDB}).
      
    \item {\bf Independent:} S'ha optat per separar les dades de la funcionalitat, podent crear implementacions diferents per les mateixes dades, fent així que fins i tot puguem substituir tot el mòdul sencer de renderització (el motor gràfic) sense haver de tocar una línia de la definició de les entitats.
      
    \item {\bf Multi-plataforma de forma nativa:} S'ha optat a que el programa s'executi a la màquina virtual de Java i usar únicament llibreries escrites totalment amb Java o, en cas necessari, amb codi natiu multi-plataforma. S'intenta el màxim possible que no s'hagi de fer codi dependent de la plataforma.
      
    \item {\bf Permetre un desenvolupament ràpid un cop estiguin fets els components bàsics:} S'ha intentat que la creació d'entitats i la seva interacció sigui el més senzilla possible.
      
    \item {\bf Fàcilment paral·lelitzable:} S'han afegit expressament comportaments indefinits del programa, especialment en l'àmbit de l'ordre d'execució de certs elements, per exemple, l'ordre en que un sistema actualitza les entitats, o l'ordre en que 2 sistemes no relacionats s'actualitzen. D'aquesta manera es permet que el llenguatge mateix creï diferents fils d'execució i els balancegi de forma òptima.
  \end{itemize}
  
\section{Definicions pròpies del llenguatge Quadriga}

  Aquí s'explicaran les definicions bàsiques de conceptes propis del Sistema d'entitats Quadriga. Per a més informació consultar \cite{EntityWikiB} o \cite{Martin07}, llocs d'on s'ha tret la inspiració bàsica per crear aquest sistema.

  A continuació es detallen les definicions, element a element. Per veure'n exemples més detallats, consultar l'apèndix \ref{chap:llenguatgeQ}.
  
  \begin{itemize}
    \item{\bf Entitat:}
      En un nivell molt concret, una entitat és simplement un enter, o identificador, únic. A nivell més abstracte, una entitat és tot aquell objecte que apareix d'alguna manera en un joc.
      
      En general, qualsevol element del joc pot ser una entitat. Des de l'escenari, o parts de l'escenari, a l'avatar del jugador, cada enemic o PNJ (personatge no jugador), ítems, etc... Fins i tot poden haver entitats més abstractes, que creïn els programadors, com una entitat que representi un nivell, o la GUI ({\em Graphical User Interface}), o fins i tot un diàleg.

      Quadriga no permet definir un tipus de dades \"{}Entitat\"{} en sí, sinó ja la té definida com a tipus bàsic i permet d'instanciar-ne de noves, eliminar-les, afegir i treure components i aplicar-li prototips (explicats més endavant).
      
      També és important introduir la idea de que les entitats formen una jerarquia en forma d'arbre. Per exemple podem crear un personatge capaç de dur diverses armes. Sabem que les armes sempre s'hauran de veure a la mà d'aquest personatge, per tal de fer-ho, indiquem que l'arma duta pel personatge és filla d'aquest a la jerarquia.
      
      
    \item{\bf Component:}
      Un component és un conjunt de dades que defineixen l'estat d'una entitat. Hi ha diferents tipus de components i cada un aporta diferents dades. Cada component també du associada certa funcionalitat, però aquesta mai es troba dintre del mateix component, sinó en un sistema (tenint sempre una separació de codi i dades, molt semblant l'esquema {\em Model-View-Controller}).
      
      Com a exemples tindríem: Transformació de l'objecte (la translació i rotació que un objecte té respecte l'origen del món), Objecte animat (component que indica quin model s'ha d'utilitzar per a renderitzar l'entitat, que a més pot contenir informació de quines animacions té, etc...), Component de IA que indicaria quines coses sap l'entitat tal que el sistema o mòdul de IA pot fer servir per a prendre decisions, etc.
      
      Per a definir un component, Quadriga permet d'assignar-li un identificador i la llista de dades que contindrà. Addicionalment es permet de llistar requisits, o sigui, altres components que prèviament una entitat necessita tenir per tal de poder tenir aquest.
      
    \item{\bf Event:}
      Un event, o esdeveniment, és un objecte que serveix per a passar missatges entre entitats. Aquests events són llençats quan succeeix alguna cosa (un event d'input, o un sistema que en llença un) i pot ser tractat per un o més sistemes. Els events es poden llençar a una entitat concreta o fer un broadcast a totes les entitats.
      
      Exemples en tindríem: cada possible ordre que el jugador dóna al joc; com moure's cap a endavant, disparar, activar un ordinador, etc. També podríem tenir events per indicar-li a una entitat que ha estat impactada per una bala, o que li diguessin que ha mort, de tal manera que es pugui activar la animació corresponent, etc.

      Un event es declara a Quadriga de forma molt similar a un Component. Només cal indicar-ne el nom i quina informació aportarà.

    \item{\bf Sistema:}
      Un Sistema és un objecte opac que s'encarrega d'una part de la funcionalitat del joc, per exemple:
      \begin{itemize}
        \item{\bf Physics System} s'activa cada 17ms i itera sobre totes les entitats amb física i fa un frame de simulació.
        \item{\bf Rendering System} itera sobre totes les entitats amb representació 2D/3D i les renderitza per pantalla.
        \item{\bf Script System} itera sobre totes les entitats que tenen un script associat i crida aquest script.
      \end{itemize}

      Un sistema sap sobre quines entitats ha d'iterar en saber quins components té associada cada entitat. Per exemple el sistema de física iterarà sobre totes aquelles entitats que tinguin algun component que descrigui informació física, i no tocarà cap altra entitat.

      La creació d'un sistema en Quadriga és la més complexa: cal especificar-li un identificador i sobre quin tipus d'entitats actua, o sigui, quins components han de tenir entitats. Després permet declarar un seguit de funcions que entren en quatre categories bàsiques:
      
      \begin{enumerate}
        \item {\bf Creació i destrucció d'entitats:} Es cridaran cada cop que una entitat aparegui per primera o última vegada, respectivament, entre les entitats afectades per aquest sistema.
        \item {\bf Actualització d'entitats:} Es cridarà cada tick del joc.
        \item {\bf Events:} Permet especificar una funció per cada tipus d'Event, que es cridarà cada vegada que una entitat pertanyent al sistema sigui afectada per aquest event.
        \item {\bf Inicialització i neteja del sistema:} Es cridaran una vegada a l'inici i final de l'execució del programa.
      \end{enumerate}
      
    \item{\bf Thread:}
      Un Thread és una agrupació de sistemes que s'han d'executar en un ordre establert, seguint així unes garanties.
      
      Normalment agrupen els sistemes de formes temàtiques: Thread de IA, Thread de física, Thread de renderitzar, Thread de lògica, etc.
      
      Per especificar un Thread en Quadriga cal indicar-ne el nom, quins sistemes contindrà (l'ordre és important) i a més permet crear mètodes d'inicialització i finalització que es cridaran a l'inici i al final de l'execució del programa.

    \item{\bf Prototip:}
      Un prototip és una eina que ajuda a crear entitats. Conceptualment és semblant a una funció que insereix els components necessaris amb els paràmetres que toquen per crear-la. És, com el seu nom indica, el prototip d'un tipus d'entitats.
      
      Un exemple força típic és tenir un prototip per a cada tipus d'enemic del joc, i a l'hora de crear els enemics d'un nivell s'indica de quin tipus són i s'afegeix informació concreta de cada instància: com la posició, el grup al qual pertanyen, etc. És molt important remarcar que els prototips només serveixen per crear entitats i aquestes no recorden quin prototip les ha instanciat.
      
      La creació d'un prototip en Quadriga és similar a la creació d'una funció, on s'especifica component a component com es construeix una entitat. A més permet de crear altres entitats \"{}filles\"{}, tal que es pugui crear un subarbre d'entitats cridant un sol prototip.

    \item{\bf Main:}
      L'Objecte Main defineix quins threads executa l'aplicació i una funció d'inicialització de la mateixa. Quan s'executa un programa Quadriga, s'ha d'especificar el nom d'un objecte tipus Main per a executar.
    
      Un Main es declara a Quadriga indicant quins Threads ha de crear, i afegint funcions d'inicialització i finalització.
    
  \end{itemize}

\section{Model de Dades}

  Un sistema d'entitats acaba sent molt semblant a una Base de Dades relacional, on cada tipus de component equival a una taula, cada entitat a un identificador i en cada fila d'aquestes taules hi ha les dades corresponents. A més, aquesta base de dades ha de complir un seguit d'especificacions.
  
  \subsection{Especificacions}
  \label{subsec:Especificacions}
  
  \begin{itemize}
    \item Els sistemes d'entitats consten de 4 elements bàsics: entitats, components, sistemes i events.
    
    \item Cada entitat és única i té un identificador únic. A més, pot tenir informació de Debug.
      \begin{itemize}
        \item Cada entitat pot tenir una entitat Pare associada, de manera que es pugui fer un graf amb les entitats (que sempre tindrà forma d'arbre).
        \item Cada entitat pot tenir un nom associat.
        \item Cada parella pare-nom equivaldrà a una única entitat.
      \end{itemize}
    \item Cada component tindrà el seu nom i un identificador únic.
    
    \item Cada sistema tindrà el seu nom i un identificador únic.
    
    \item Cada event tindrà el seu nom i un identificador únic.
    
    \item Cada sistema s'ha d'associar amb els components i events als quals afecta.
    
    \item Cada sistema haurà de guardar un registre de totes les entitats a les que ha afectat en la darrera iteració del joc.
    
    \item Quan un component s'associa a una entitat, es crea una instància d'aquest component i se'n guarden les dades.
    
    \item Cada tipus de component té unes dades associades diferents que es programen mitjançant el llenguatge Quadriga.
  \end{itemize}
  
  \subsection{Implementació}

  L'esquema Entiat-Relació resultant es mostra a la figura \ref{fig:EntitatRelacio}. I la base de dades relacional tindria la forma mostrada a la figura \ref{fig:BBDDRelacional}.

  \begin{figure}
    \includegraphics[width=1\linewidth]{./img/EntitatRelacio.png}
    \caption{Model Entitat-Relació \label{fig:EntitatRelacio}}
  \end{figure}

  \begin{figure}
    \includegraphics[width=1\linewidth]{./img/BBDDRelacional.png}
    \caption{Base de dades. \label{fig:BBDDRelacional}}
  \end{figure}
  
  Cal destacar en aquest apartat, que el model serà lleugerament diferent per cada programa compilat amb Quadriga, doncs es generarà una taula addicional per a cada component descrit, amb els camps corresponents. No serà la mateixa base de dades per un joc simple, estil {\em Tetris}, que per un MMORPG estil {\em World of Warcraft}, tal i com la intuïció ens fa saber.
  
  La implementació d'aquesta base de dades s'ha optat per fer-la amb un motor SQL per a Java anomenat {\em HSQLDB}. Això ens proporciona eines com les cerques optimitzades (per exemple, cercar totes les entitats que continguin un seguit de components, o saber a quins sistemes cal enviar un event) i facilitats com guardar i carregar la base de dades de forma molt senzilla, però a canvi es perd força rendiment respecte si la implementació del model de dades es fes directament respecte a les especificacions esmentades a la secció \ref{subsec:Especificacions}.

\section{Editar el llenguatge}

  Per tal de facilitar l'edició de {\em Quadriga} s'ha implementat un senzill plug-in per a {\em Notepad++} que es pot descarregar des de \url{http://code.google.com/p/quadriga/downloads/detail?name=quadriga.xml} i importar al mateix {\em Notepad++}. Aquest programa és un simple editor de text que ampliem d'aquesta manera per a que ressalti paraules claus de Quadriga, no aporta funcions de completat automàtic ni marcat d'errors.

\chapter{Resultats}
\label{chap:Resultats}

  Per tal de validar el sistema funcionava, s'ha provat de fer un joc senzill amb ell, en aquest cas, un clon del Tetris.
  
  El Tetris és un joc creat el 1984, de mecàniques similars a un puzzle i molt conegut arreu del món. La seva simplicitat i la seva fama el fan ideal com a exercici de programació. En podem veure una imatge a la figura \ref{fig:ImatgeTetris}.
  
  \begin{figure}
    \centering
    \includegraphics[width=0.5\linewidth]{./img/ImatgeTetris.png}
    \caption{Captura de pantalla del joc final \label{fig:ImatgeTetris}}
  \end{figure}
  
  Les mecàniques del Tetris són molt senzilles: distingim dues entitats bàsiques, el taulell i la peça. La peça cau des de la part superior del taulell, i el jugador pot modificar-ne la trajectòria i girar-la. Un cop la peça no pot descendre més, aquesta queda fixada al taulell i apareix una altra peça a la part superior. Si no pot apareixer aquesta peça, el joc finalitza. Al fixar una peça, pot ser que s'ocupi una línia sencera, en aquest cas s'elimina la línia i es beneficia al jugador amb un seguit de punts.

  En aquest capítol farem un repàs primer dels elements de Quadriga que s'han creat per programar el Tetris, seguidament es farà un cop d'ull a uns elements \"{}estàndard\"{} que s'han creat per tal d'ajudar-ne al desenvolupament. Finalment es farà un anàlisi sobre els objectius de disseny, punt per punt.
  
  \section{Elements propis del Tetris}

    Per tal de veure visualment com està muntat aquest tetris, s'ha fet un diagrama semblant a l'UML de la seva estructura a la figura \ref{fig:TetrisEntitats}. També s'ha confeccionat una llegenda per entendre-ho millor que es pot veure a la figura \ref{fig:GuiaDiagramaQuadriga}.

    \begin{figure}
      \includegraphics[width=1\linewidth]{./img/TetrisEntitats.png}
      \caption{Esquema dels elements que formen el tetris \label{fig:TetrisEntitats}}
    \end{figure}

    \begin{figure}
      \centering
      \includegraphics[width=0.7\linewidth]{./img/GuiaDiagramaQuadriga.png}
      \caption{Llegenda \label{fig:GuiaDiagramaQuadriga}}
    \end{figure}

    \subsection{Entitats / Prototips}
      
      \begin{itemize}
        \item {\bf Tetris}
          Representa l'estat actual del taulell de joc, amb els cubs posats, així com també conté informació sobre el nivell actual, els punts i les línies aconseguides pel jugador.
          
        \item {\bf Peça}
          Representa la peça que el jugador ha de col·locar en cada moment. Seria el més semblant a l'Avatar del jugador.
          
        \item {\bf CubDeMarc}
          Representa cada un dels cubs que fan de marc del taulell de joc.
          
        \item {\bf CubDeTaulell}
          Representen els cubs interiors del taulell, tant els col·locats com els que la peça actual aporta. Es diferencien dels anteriors ja que aquests han de tenir un color diferent depenent de la peça.
          
      \end{itemize}

    \subsection{Components}

      \begin{itemize}
        \item {\bf Taulell}
          Guarda informació de quines caselles del taulell estan ocupades per un cub, i en guarda una referència per eliminar-los o moure'ls quan el jugador completa una línia.
          
        \item {\bf Puntuació}
          Guarda informació sobre el nivell, les línies completades i la puntuació feta.
          
        \item {\bf Peça}
          Guarda informació sobre la forma i posició de la peça que actualment cau.
          
        \item {\bf ColorCub}
          Guarda informació sobre el color d'un cub.
          
        \item {\bf GameOverScreen}
          Representa la pantalla final, quan el jugador s'ha rendit o ha perdut.
          
      \end{itemize}

    \subsection{Events}

      \begin{itemize}
        \item {\bf NovaPeça}
          Es crea una nova peça, ja sigui per que s'inicia el joc o s'ha col·locat l'anterior. Aporta informació sobre quin tipus de peça s'ha de crear.
          
        \item {\bf InserirPeça}
          La peça ha arribat a sota i s'ha d'inserir al taulell.
          
        \item {\bf Left, Right, Down, TurnL, TurnR}
          El jugador ha donat instruccions per moure la peça o girar-la.
          
        \item {\bf Escape}
          El jugador ha premut la tecla d'escapament, volent finalitzar el joc o programa.
          
      \end{itemize}

    \subsection{Sistemes}

      \begin{itemize}
        \item {\bf LògicaTaulell}
          Controla la inserció de peces, l'acumulació de punts i si s'han completat línies, augmentant el nivell si s'escau.
          
        \item {\bf LògicaPeça}
          Controla que una peça es mogui segons les ordres del jugador i caigui a una velocitat donada pel nivell actual. També comprova si ha arribat al final i cal inserir-la.
          
        \item {\bf GameOverScreen}
          Un cop el jugador ha acabat la partida, en mostra la puntuació i espera que el jugador doni ordres de tancar el programa.
          
        \item {\bf InputManager}
          Vigila quin input dona el jugador i crea els events corresponents.
          
      \end{itemize}
      
    \subsection{Threads}
    
      \begin{itemize}
        \item {\bf ThreadTetris}
          Inicialitza el joc i executa els 4 sistemes anteriors.
          
      \end{itemize}
    
  \section{Llibreria SimpleRender}

    \subsection{Entitats / Prototips}

      \begin{itemize}
        \item {\bf TextRenderer}
          Renderitza un text donada una {\em Font}, una posició i un {\em String}.
      \end{itemize}


    \subsection{Components}

      \begin{itemize}
        \item {\bf Transform}
          Guarda informació sobre la translació, rotació i escala d'una entitat.
          
        \item {\bf SceneComponent}
          Marca un objecte com a arrel de l'escena.
          
        \item {\bf SimpleCamera}
          Guarda informació sobre la càmera de l'escena.
          
        \item {\bf ColorMaterial}
          L'objecte es renderitza amb un material de color pla.
          
        \item {\bf BoxRender}
          L'objecte es renderitza com una caixa.
          
      \end{itemize}
      
    \subsection{Threads}
    
      \begin{itemize}
        \item {\bf SimpleRenderThread}
          Renderitza tots els objectes de l'escena que contenen algun component \"{}renderitzable\"{} com {\em BoxRender}.
          
      \end{itemize}
      
\section{Objectius de disseny}

  S'analitza, punt per punt, si s'han complert els objectius de disseny del programa.
    
  \begin{figure}
    \centering
    \includegraphics[width=0.5\linewidth]{./img/ImatgeUbuntu.png}
    \caption{Captura de pantalla del joc sobre Ubuntu \label{fig:ImatgeUbuntu}}
  \end{figure}

  \begin{itemize}
    \item {\bf Open Source} \hfill \\
      El codi es pot trobar a \url{https://code.google.com/p/quadriga/} sota llicència {\bf GNU Lesser GPL}.
      
    \item {\bf Independent} \hfill \\
      En l'exemple s'usen unes quantes llibreries Java per implementar aspectes importants del joc. {\em OpenGL} i l'{\em input} del jugador s'obtenen a partir de {\bf LWJGL}. Per fer-les servir des de {\em Quadriga} cal emprar {\em Components} que anomenaríem "estàndard" sota el paquet {\em cat.quadriga.base}. Fer aquests components no és senzill, però es podrien substituir els sistemes si es volgués fer l'esforç sense masses problemes.
      
    \item {\bf Multi-plataforma de forma nativa} \hfill \\
      Com es veu a la figura \ref{fig:ImatgeUbuntu}, el joc corre sobre {\em Ubuntu} sense cap diferència. El codi Java i Quadriga no s'han tocat, però si que cal fer una petita modificació a l'hora de crear l'executable, doncs {\em OpenGL} necessita accedir a llibreries natives que són diferents en cada plataforma.
      
    \item {\bf Permetre un desenvolupament ràpid un cop estiguin fets els components bàsics} \hfill \\
      El codi final són 2 arxius, un de lògica de unes 800 línies i un altre de input de 65. És de fet quasi més llarg fer el disseny que no implementar-lo.
      
      Addicionalment s'ha provat d'afegir funcionalitat. En vint minuts s'ha aconseguit inserir una guia per indicar la peça següent a caure. En total s'han afegit 140 línies (afegir el component, prototip i sistema que controlin la peça següent), de les quals pràcticament totes són una còpia de codi ja fet, i s'han hagut de modificar 3 punts del joc, on s'instanciava una nova peça per tal de fer que agafés el tipus de peça que hi havia guardat. Veure figura \ref{fig:ImatgePecaSeguent}.
      
    \item {\bf Fàcilment paral·lelitzable} \hfill \\
      Actualment el programa conté una opció de fer córrer cada {\em Thread} en paral·lel, però l'opció teòricament òptima de paral·lelitzar l'actualització de cada entitat sobre cada sistema (aconseguint una paral·lelització no sobre el nombre de threads, sinó sobre el nombre d'entitats) no s'ha implementat.
  \end{itemize}
    
  \begin{figure}
    \centering
    \includegraphics[width=0.5\linewidth]{./img/ImatgePecaSeguent.png}
    \caption{Captura de pantalla del joc amb modificacions \label{fig:ImatgePecaSeguent}}
  \end{figure}

  Per tal de validar els resultats, s'ha considerat que el fet de poder-hi programar-hi un joc consisteix una prova força complerta de que el projecte funciona. S'ha demostrat que el joc es pot ampliar fàcilment, ara es proposaran alguns dissenys de millora:
  
  \begin{itemize}
    \item {\bf Multijugador a pantalla dividida:} Per tal d'implementar aquesta millora, cal duplicar les entitats principals: tenir 2 sistemes i 2 peçes. També cal ampliar els events d'ordres del jugador tal que aportin informació de quin jugador l'ha feta, i que les funcions que responen a aquests events actuin sobre la peça adequada.
    
    \item {\bf Multijugador en xarxa:} Aquesta ampliació és en part més complexa que l'anterior, ja que implica crear un módul de xarxa. Un cop creat aquest módul, les modificacions al programa Quadriga són mínimes, ja que només cal fer que els events que el jugador local crea s'enviin al remot, i fer que els events remots es tradueixin als events locals.
    
    \item {\bf Crear un fons de pantalla més estètic:} Cal crear un component i entitat que ho controlin, així com ampliar la llibreria gràfica. Tot això aniria associat a la llibreria estàndard de Quadriga i les modificacions des del nucli de la lògica serien mínimes.
    
    \item {\em Crear un menú principal, amb pantalla de rècords:} Caldria crear una màquina d'estats que implementés aquest menú. La llibreria estàndard ja incorpora utilitats per renderitzar qualsevol text en {\bf Unicode}. Adicionalment, es podria afegir funcionalitat per guardar l'estat de la base de dades i així mantenir els rècords entre partides.
  \end{itemize}

  

\chapter{Conclusions i Millores}
\label{chap:Conclusions}


\begin{itemize}
  \item S'ha dissenyat un llenguatge de programació que segueix el paradigma dels {\em Sistemes d'Entitats} que s'ha anomenat Quadriga.
  \item S'ha dissenyat i implementat un {\bf compilador} i un {\bf entorn d'execució} per al prototipatge i/o programació de videojocs.
  \item S'ha creat una base de dades creada dinàmicament que suporta qualsevol tipus de joc sobre un format estandaritzat.
  \item S'ha creat un subratllador de sintaxi per facilitar la seva programació.
  \item S'ha implementat un joc d'exemple basat en el clàssic {\bf Tetris}.
  \item S'ha desenvolupat una petita llibreria bàsica per a renderitzar textos i alguns formes geomètriques simples: cubs i esferes.
\end{itemize}

Com a {\bf incidències} importants cal destacar:

\begin{itemize}
  \item La idea d'usar una base de dades {\em SQL} no ha donat resultats massa bons, ja que consumeix molts recursos. Tot i això, es pot fer servir amb poques modificacions per tal de guardar l'estat del joc (funcionalitat de partida guardada) i carregar-lo de forma molt simple.
\end{itemize}

Com a possibles {\bf millores} que aportar al projecte:

\begin{itemize}
  \item Fer que el compilador generi bytecode per executar directament a la màquina virtual de Java, millorant el rendiment dels programes desenvolupats a Quadriga.
  \item Crear una implementació específica del model de dades, sense usar una base de dades SQL. Així el rendiment milloraria considerablement.
  \item Ampliar la llibreria estàndard per incloure la següent funcionalitat:
  \begin{itemize}
    \item Renderitzar formes geomètriques arbitràries (actualment només se suporten cubs, esferes i textos).
    \item Crear una manera de renderitzar models animats per esquelet.
    \item Ampliar els materials per a tenir efectes més complexos.
    \item Afegir funcionalitat de so.
    \item Afegir funcionalitat de joc en xarxa.
  \end{itemize}
  D'aquesta manera el programador o dissenyador tindria més facilitat per desenvolupar-hi jocs.
  \item Crear un plug-in de la {\bf IDE} {\em Eclipse} per a desenvolupar i debugar més fàcilment.
\end{itemize}



\begin{onehalfspace}
\bibliographystyle{ieeetr}
\bibliography{referencies}
\addcontentsline{toc}{chapter}{Bibliografia}
\end{onehalfspace}

\appendix
\newpage
\chapter{El llenguatge Quadriga}
\label{chap:llenguatgeQ}

\section{Estructura dels programes}

Un programa fet amb Quadriga consta de diferents elements (Component, Sistema, Event, Thread, Prototip, Main) distribuïts en diferents fitxers o paquets. El concepte de paquets està directament agafat de {\em Java} i serveix per a crear espais de noms, tal que un programador sigui lliure de crear elements amb el nom que vulgui, que sempre seran compatibles amb altres amb el mateix nom, però diferent paquet.

Cada paquet de Quadriga està representat per un fitxer. Els fitxers de quadriga acaben en {\em \"{}.qdg\"{}}. Si un paquet es diu {\em \"{}tetris.input\"{}} aleshores el fitxer serà anomenat {\em \"{}tetris/input.qdg\"{}}. Aquesta metodologia permet al compilador trobar fàcilment els arxius que necessita. En quadriga no existeix cap ordre per incloure fitxers externs. En comptes d'això, cal declarar els elements que es necessiten, i el compilador els buscarà als directoris d'inclusió. Si per exemple volem fer servir el {\em Component Transform} del paquet {\em cat.quadriga.base} aleshores hem de declarar {\em @component cat.quadriga.base.Transform} abans de fer-lo servir, i el compilador ja s'encarregarà de trobar-lo. Aquesta sentència es pot usar per predeclarar un element del mateix fitxer abans d'on està pròpiament escrit per tal d'utilitzar-lo, de forma similar a com ho fa {\bf C}. Per tal d'importar una classe de {\em Java} cal usar la paraula clau {\bf @java} tal com s'importen la resta d'elements. Com a java, es considera que les classes del paquet {\em java.lang} estan automàticament importades.

El llenguatge utilitza essencialment la mateixa estructura que Java, amb la diferència que a Quadriga no es creen classes, sinó els elements propis: {\em Components, Sistemes, Events, Threads, Prototips i Mains}. També s'han afegit algunes instruccions extres. Totes les paraules clau pròpies de {\em Quadriga} comencen amb el símbol {\bf @}.

\subsection{Entitat}

\begin{verbatim}
Entity<Component1, Component2, ...>
\end{verbatim}

Una entitat, en Quadriga, és simplement un tipus de variable o paràmetre. Com que cada entitat està identificada pels seus components, es permet de nombrar quins components se suposa que té associats de forma similar a com es fan els {\em Generics} en Java.

\section{Declaració de tipus de dades}

\subsection{Components}

\begin{verbatim}
@component Puntuació {
  *CTaulell;
  int punts  = 0;
  int línies = 0;
  int nivell = 0;
}
\end{verbatim}

La declaració d'un Component comença amb la paraula clau {\em \"{}@component\"{}} seguida de l'identificador desitjat del component. Seguidament, dintre de claus (\{ \}), es declaren les variables d'estat de forma similar a Java, amb la possibilitat d'establir un valor inicial si es desitja, tal com es mostra en l'exemple.

També dintre de les claus, i a l'inici es pot especificar quins Components necessita aquest Component (requisits previs), afegint un asterisc {\bf *} i el nom d'aquests components.

Els tipus permesos són els tipus primitius de Java (byte, short, int, long, float, double, boolean, char) i classes que siguin {\em Serialitzables}.

\subsection{Events}

\begin{verbatim}
@event NovaPeça {
  int tipus;
}
\end{verbatim}

Un event es declara de forma idèntica a un component, a excepció de la paraula clau inicial {\em \"{}@component\"{}} que és substituïda per {\em \"{}@event\"{}}; i no hi ha \"{}requisits\"{}.

\subsection{Sistemes}

\begin{verbatim}
@system LògicaPeça {
  @components {
    CPeça
  }
  
  @init {...}
  
  @cleanUp {...}
  
  @update(Entity<CPeça> entitat: ENTITY, float dt : DELTA_TIME) {...}
  
  @newEntity(Entity<CPeça> entitat: ENTITY) {...}
  
  @eraseEntity(Entity<CPeça> entitat: ENTITY) {...}
  
  @event NovaPeça(
                  NovaPeça event: EVENT, 
                  Entity<CPeça> entitat: ENTITY
                  )
  {...}
  
  @event ELeft(Entity<CPeça> entitat: ENTITY) {...}
  
  @event ERight(Entity<CPeça> entitat: ENTITY) {...}
  
  @event EDown(Entity<CPeça> entitat: ENTITY) {...}
  
  @event ETurnL(Entity<CPeça> entitat: ENTITY) {...}
  
  @event ETurnR(Entity<CPeça> entitat: ENTITY) {...}
}
\end{verbatim}

La declaració d'un Sistema comença de forma similar a la d'un component, amb la paraula clau {\em \"{}@system\"{}}, el nom del sistema, i els diferents elements d'aquest. El primer element que cal introduir sempre, i l'únic obligatori, és la llista de components que una Entitat necessita tenir per que aquest sistema l'afecti. Aquesta llista se situa entre les claus de la següent construcció:

\begin{verbatim}
  @components { }
\end{verbatim}

Els elements de la llista poden anar separats de punts i coma ({\bf ;}), però poden estar separats únicament per espais.

A continuació vindran les funcions del sistema. Les dues primeres a destacar són {\bf @init} i {\bf @cleanUp}. Aquestes s'executen a l'inici i a la fi de l'execució, una única vegada. Entre 2 sistemes, l'ordre de en que s'executin aquestes funcions depèn únicament de l'ordre en el qual s'han declarat en un Thread. Si 2 Threads declaren el mateix sistema, aquest s'executarà dues vegades.

La funció més important d'un sistema és normalment la funció {\bf @update}. Aquesta funció s'executa un cop per {\em tick} del joc. En el pas de paràmetres cal comentar la introducció del concepte de {\bf semàntiques}. Aquestes semàntiques s'indiquen després de cada paràmetre usant el símbol dos punts {\bf :}. El tipus de paràmetre cal que sigui compatible amb la semàntica. En el cas de la funció {\bf @update} se'n proporcionen 2:

\begin{itemize}
  \item {\bf ENTITY}: Correspon a la entitat actualitzada.
  \item {\bf DELTA\_TIME}: És el temps, en segons, des de la última vegada que es va cridar la funció per la mateixa entitat.
\end{itemize}

Les següents funcions són {\bf @newEntity} i {\bf @eraseEntity}, que es criden quan una entitat té per primera vegada els components necessaris o deixa de tenir-los, respectivament. Aquestes funcions només proporcionen la semàntica {\bf ENTITY}.

Finalment hi ha les funcions que tracten events. Aquestes funcions comencen amb la paraula clau {\bf @event}, seguida de l'identificador de l'event que es vol tractar. Quan una entitat que pertany al sistema rebi l'event, automàticament es cridarà aquesta funció. A part de la semàntica {\bf ENTITY}, es proporciona la semàntica {\bf EVENT}, que ha de tenir el mateix tipus que l'event tractat i proporcionarà totes les dades que vinguin amb aquest event.

\subsection{Prototips}

\begin{verbatim}
@prototype PPeça(Entity<Puntuació> taulell)
{
  Transform()
  CPeça(taulell: taulell)
  @child {
    "cub1" : CubDeTaulell(posX:0, posY:0, color: 1)
    "cub2" : CubDeTaulell(posX:0, posY:0, color: 1)
    "cub3" : CubDeTaulell(posX:0, posY:0, color: 1)
    "cub4" : CubDeTaulell(posX:0, posY:0, color: 1)
  }
}
\end{verbatim}

L'estructura d'un prototip és una mica més complexa. Primer cal especificar-ne el nom, seguit dels paràmetres que té el prototip. Després se li afegeixen un seguit de plantilles per a dues coses: Components que tindrà l'entitat, i altres entitats filles que crearà. Els elements sempre es creen i afegeixen en l'ordre establert.

Per a afegir un component, simplement cal escriure el nom i, entre parèntesi, donar valor als camps. Per fer-ho, només cal escriure el nom del camp, dos punts {\bf :} i el valor a assignar-hi. Aquells camps que tinguin valor per defecte no cal que siguin assignats.

Per a especificar Entitats filles, cal especificar-les dintre d'una estructura {\em\"{}@child\{ \}\"{}}. Aquí la fòrmula a utilitzar és indicar primer el nom (de forma opcional), a continuació el prototip utilitzat i finalment els paràmetres, de forma similar als Components. Si es vol sobreescriure un valor d'un Component d'algun d'aquests fills, s'haurà d'especificar el nom del component, punt {\bf .} i finalment el camp del component.

Finalment també es pot executar codi si aquest va dintre d'una estructura {\em\"{}@init\{ \}\"{}}, que pot ser repetida diverses vegades al prototip.

\subsection{Threads}

\begin{verbatim}
@thread LlogicaTetris {
  @system SGameOverScreen;
  @system InputManager;
  @system LlògicaTaulell;
  @system LlògicaPeça;
  
  @init {...}
  
  @cleanUp {...}
}
\end{verbatim}

Un Thread indica quins Sistemes s'executaran, i en quin ordre, i una funció d'Init i CleanUp (opcionals) que s'executaran a l'inici i final de l'execució del programa.

\subsection{Main}

\begin{verbatim}
@main Main {
  @thread SimpleRenderThread
  @thread LlogicaTetris
  
  @init {...}
  
  @cleanUp {...}
}
\end{verbatim}

L'objecte Main indica quins Threads s'executen i l'ordre d'aquests (si l'execució no és paral·lela), a més d'un Init i un CleanUp.

\section{Instruccions pròpies}

\subsection{Nova Entitat}

\begin{verbatim}
@newEntity(nom, pare)
\end{verbatim}

Aquesta instrucció retorna una Entitat nova. Se li pot especificar un nom i una Entitat pare, o deixar els paràmetres a null.

\subsection{Eliminar Entitat}

\begin{verbatim}
@eraseEntity entitat;
\end{verbatim}

Elimina una Entitat.

\subsection{Cercar Entitat}

\begin{verbatim}
@find[nom]
\end{verbatim}

Cerca una Entitat amb el nom especificat i sense pare.

\subsection{Aplicar prototip}

\begin{verbatim}
@prototype entitat : nomPrototip( param1: valor1, param2: valor2, 
                                  Component1.camp1: valor3, Component1.camp2: valor4,
                                  ...)
\end{verbatim}

Fa que a una Entitat se li apliqui un Prototip.

\subsection{Agregar Component}

\begin{verbatim}
@add nomComponent(param1: valor1, param2: valor2) : entitat;
\end{verbatim}

Fa que a una Entitat se li agregui un Component.

\subsection{Cridar Event}

\begin{verbatim}
@event[temps] nomEvent(param1: valor1, param2: valor2) : entitat ;
\end{verbatim}

Crida un event. El temps ha de ser positiu, o es pot ometre (per fer una crida instantània), així com la Entitat sobre la que cridar també es pot ometre (fent un Broadcast).

\subsection{Accedir a un Component}

\begin{verbatim}
entitat[Component]
\end{verbatim}

Accedeix a un Component de la Entitat.

\subsection{Accedir a una Entitat filla}

\begin{verbatim}
entitat["nomFilla"]
\end{verbatim}

Accedeix a una Entitat filla amb el nom especificat.

\subsection{Nombre Aleatori}

\begin{verbatim}
@rnd
\end{verbatim}

Retorna un objecte del tipus {\em java.util.Random}.

\section{Execució}

Per executar un programa de quadriga cal executar la classe {\em cat.quadriga.parsers.Quadriga} amb un únic paràmetre: el nom de l'objecte Main que es vol executar. Cal que Java pugui trobar totes les llibreries Java necessàries (vecmath, hsqldb, lwjgl, slick-util, lwjgl-util) com les llibreries natives de LWJGL.
\chapter{El projecte Quadriga}

El projecte, amb el codi font i alguns exemples, es pot trobar a la pàgina \url{http://quadriga.googlecode.com}. Adicionalment també es pot trobar aquesta mateixa memòria. En el moment d'escriure aquesta memòria, el projecte es pot aconseguir d'un servidor de {\bf subversion}, actualment a la revisió número 100. El codi està pensat per a ser compilat amb la {\em IDE Eclipse} de la següent manera:

\begin{enumerate}
  \item Cal instalar Eclipse amb els plugins {\em subclipse} \url{http://subclipse.tigris.org/} i {\em JavaCC Eclipse Plug-in} \url{http://eclipse-javacc.sourceforge.net/}.
  
  \item Cal descarregar el projecte {\bf SVN} des d'eclipse amb repositori a \url{http://quadriga.googlecode.com/svn/trunk/Quadriga Parser/}. \\
        Alternativament es pot descarregar tot el projecte amb {\bf SVN} ja sigui amb la consola o usant un programa com {\bf Tortoise SVN} \url{http://tortoisesvn.tigris.org/}.
        
  \item Descarregar les llibreries {\bf LWJGL} \url{http://lwjgl.org/download.php} i {\bf HSQLDB} \url{http://sourceforge.net/projects/hsqldb/files/}. 
        Incloure ambdues llibreries al projecte d'Eclipse (figura \ref{fig:incloure}).
        
  \item Finalment cal indicar-li a l'Eclipse on troba les llibreries dinàmiques de {\bf LWJGL} (figura \ref{fig:incloure3}).
\end{enumerate}

\begin{figure}
  \centering
  \subfloat[Obrir les propietats del projecte]{\label{fig:incloure1}\includegraphics[width=0.48\textwidth]{./img/incloure1.png}}                
  \subfloat[Afegir els les llibreries (jar) necessàries]{\label{fig:incloure2}\includegraphics[width=0.48\textwidth]{./img/incloure2.png}}
  \caption{Com incloure les llibreries necessàries}
  \label{fig:incloure}
\end{figure}

\begin{figure}
  \centering
  \includegraphics[width=0.58\linewidth]{./img/incloure3.png}
  \caption{Indicar-li el path a les llibreries dinàmiques \label{fig:incloure3}}
\end{figure}

\newpage
\pagestyle{empty}

\begin{onehalfspace}

{\bf RESUM:} Aquest projecte tracta del desenvolupament d'una plataforma on crear la lògica de videojocs. S'ha decidit crear un llenguatge especial per a descriure aquesta lògica i per fer-ho s'ha seguit l'arquitectura del \"{}Sistemes d'Entitats\"{}, ja que és una arquitectura que s'ha anat desenvolupant al llarg dels anys dins del mateix sector dels videojocs. Per a fer-ho s'ha creat un analitzador sintàctic i un generador de codi, així com un intèrpret i un entorn d'execució. Addicionalment s'hi ha afegit una llibreria que proveeixi els altres mòduls necessaris per desenvolupar un videojoc i s'ha implementat el Tetris a mode de prova i exemple.
\\

{\bf RESUMEN:} Este proyecto trata del desarrollo de una plataforma donde crear la lógica de videojuegos. Se ha decidido crear un lenguage especial para describir esta lógica y para ello se ha usado la arquitectura de \"{}Sistema de Entidades\"{}, ya que es una arquitectura que se ha desarrollado a lo largo del tiempo dentro de la misma industria de los videojugos. Para ello se ha creado un analizador sintáctico y un generador de código, además de un intérprete y un entorno de ejecución. Adicionalmente se ha añadido una librería que provea los otros módulos necesarios para el desarrollo de un videojugo y se ha implementado el Tetris a modo de prueba y ejemplo.
\\

{\bf ABSTRACT:} This project is about the development of a platform where one can create the logic of a video game. To do so, a language has been created to describe this logic using the architecture known as \"{}Entity System\"{}, because this architecture was developed in the video game industry itself. In order to accomplish that, a parser and a code generator have been created, also a code interpreter and an execution environment have been added. In order to have all the needed functionality to create a video game, a standard library has been developed and a Tetris Clone was created as a test and example.

\end{onehalfspace}


\end{document}
