\chapter{Conclusions i Millores}
\label{chap:Conclusions}


\begin{itemize}
  \item S'ha dissenyat un llenguatge de programació que segueix el paradigma dels {\em Sistemes d'Entitats} i s'ha anomenat Quadriga.
  \item S'ha dissenyat i implementat un {\bf compilador} per aquest llenguatge.
  \item S'ha implementat un {\bf entorn d'execució} que proporcioni al programa tots els elements necessaris per al seu correcte funcionament en qualsevol plataforma suportada per Java.
  \item S'ha creat una base de dades especificada dinàmicament (resultat de la compilació) que suporta qualsevol tipus de joc sobre un format estandaritzat.
  \item S'ha creat un subratllador de sintaxi per facilitar la seva programació.
  \item S'ha implementat un joc d'exemple basat en el clàssic {\bf Tetris}.
  \item S'ha desenvolupat una petita llibreria bàsica per a renderitzar textos i algunes formes geomètriques simples: cubs i esferes.
  \item S'ha dissenyat un model de llibreria tal que sigui fàcil de transportar a altres plataformes (fins i tot plataformes mòbils) emprant les últimes tècniques de renderitzat (mitjançant els {\em Shaders} d'OpenGL 2.0).
\end{itemize}

Com a {\bf incidències} importants cal destacar:

\begin{itemize}
  \item La idea d'usar una base de dades {\em SQL} no ha donat resultats massa bons, ja que consumeix molts recursos. Tot i això, es pot fer servir amb poques modificacions per tal de guardar l'estat del joc (funcionalitat de partida guardada) i carregar-lo de forma molt simple.
\end{itemize}

Com a possibles {\bf millores} a aportar al projecte:

\begin{itemize}
  \item Fer que el compilador generi bytecode per executar directament a la màquina virtual de Java, millorant el rendiment dels programes desenvolupats a Quadriga.
  \item Crear una implementació específica del model de dades, sense usar una base de dades SQL. Així el rendiment milloraria considerablement.
  \item Ampliar la llibreria estàndard per incloure la següent funcionalitat:
  \begin{itemize}
    \item Renderitzar formes geomètriques arbitràries (actualment només se suporten cubs, esferes i textos).
    \item Crear una manera de renderitzar models animats per esquelet.
    \item Ampliar els materials per a tenir efectes més complexos.
    \item Afegir funcionalitat de so.
    \item Afegir funcionalitat de joc en xarxa.
  \end{itemize}
  D'aquesta manera el programador o dissenyador tindria més facilitat per desenvolupar-hi jocs.
  \item Crear un plug-in de la {\bf IDE} {\em Eclipse} per a desenvolupar i debugar més fàcilment.
\end{itemize}