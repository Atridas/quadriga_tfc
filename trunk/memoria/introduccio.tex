\newpage
\section{Introducció}

Des de l'aparició dels primers videojocs, la seva evolució ha estat sempre emparellada amb una evolució tecnològica constant i, en certs punts, accelerada. Aquesta evolució tecnológica no només consta de millores en el rendiment i qualitat de cada joc sinó que, a més, s'ha aconseguit generelitzar solucions i aplicar tecnologia d'un videojoc a altres i fins hi tot crear eines que permeten el desenvolupament de jocs a partir d'elles.

Si mirem aquesta evolució des de més a prop, veiem com al principi els videojocs eren evolucions d'altres jocs típics de bar, màquines escurabutxaques com el pinball seriren d'inspiració per a crear els primers videojocs comercials, amb un model de negoci molt semblant. Posteriorment, la necessitat de crear grans quantitats de jocs va obligar als desenvolupadors a crear màquines que poguéssin executar més d'un joc i així abaratir costos i finalment, amb l'arribada dels ordinadors personals i les consoles de sobretaula, es comença a desenvolupar un mercat per a jocs sobre plataformes genèriques.
\\

Poc a poc van apareixent els motors gràfics - programes o móduls encarregats del renderitzat d'un joc o d'un programa amb gràfics 2D o 3D - o fins hi tot motors de joc -una plataforma per desenvolupar-hi un joc a sobre -. En un principi aquests motors s'utilitzaven dintre de la mateixa companyia que el creava. Per exemple LucarArts creà {SCUMM} ({Script Creation Utility for Manic Mansion}) a l'hora que creava la seva aventura gràfica de "Point \& Click" Manic Mansion. Aquest mateix programa fou utilitzat després en altres jocs com Indiana Jones i l'Última Creuada, LOOM, El Dia del Tentacle i tres jocs de la saga Monkey Island.

Un pas més endavant el va dur Id Software amb el seu id Tech. Aquest motor - i les seves evolucions - no només es feu servir per fer jocs com Doom i Doom II d'Id Software, sinó que es va vendre a altres companyies per a fer altres jocs, tot i que aquests jocs serien molt semblants al Doom original. Posteriorment fins hi tot hi va haver companyies que basaven el seu negoci no en vendre jocs, sinó en vendre motors a altres companyies que els féssin servir; és doncs l'aparició definitiva dels motors com a Middleware.
\\

\subsection{Els sistemes d'entitats}

Paral·lelament a l'evolució tecnológica ja esmentada, la metodologia de desenvolupament i els mateixos llenguatges de programació han anat evolucionant. L'evolució més important fou quan es va passar de programar bàsicament en {\bf C} a {\em C++}. El canvi de paradigma, però, s'ha demostrat difícil i, tot i que l'ús de la metodologia orientada a objectes és predominant a quasi totes les àrees d'un motor, encara n'hi ha alguna on porta problemes.
\\

El cas més important, i el que en aquest treball ens centrarem és en la definició de la llògica d'un joc.

Quan un programador de {C++} o qualsevol llenguatge orientat a objectes s'asseu davant el desafiament de programar la llògica d'un joc, trenca els diferents objectes que el poblen i els distribueix en diferents grups i subgrups, després en programa les funcionalitats comunes i acaba creant una jerarquia de classes que defineixen tots els objectes i les seves interrelacions. 
Aquesta aproximació sembla senzilla, però acaba comportant diversos problemes; per exemple, a vegades el trencament en grups no sembla evident. Posem per exemple que volem tenir en el nostre joc vehicles, i que n'hi ha de terrestres (Cotxe, Tanc) i de marítims (Lanxa, Moto d'aigua). La nostra primera aproximació ens crearà una gerarquia de tres nivells, amb els Vehicles agrupant-ho tot, on després distingim entre terrestres i marítims i finalment tenim cada tipus concret de vehicle. Mesos dintre de la producció del joc se'ns demana que afegim un nou vehicle amfibi, on el col·loquem?