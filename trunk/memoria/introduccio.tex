\chapter{Introducció}

Des de l'aparició dels primers videojocs, la seva evolució ha estat sempre emparellada amb una evolució tecnològica constant i, en certs punts, accelerada. Aquesta evolució tecnològica no només consta de millores en el rendiment i qualitat de cada joc sinó que, a més, s’han aconseguit generalitzar solucions i aplicar la tecnologia d’un videojoc a d'altres i fins i tot crear eines que permeten el desenvolupament de jocs a partir d'elles.

Si mirem aquesta evolució des de més a prop, veiem com al principi els videojocs eren evolucions d'altres jocs típics de bar. Màquines escurabutxaques com el pinball serviren d'inspiració per a crear els primers videojocs comercials, amb un model de negoci molt semblant. Posteriorment, la necessitat de crear grans quantitats de jocs va obligar als desenvolupadors a crear màquines que poguessin executar més d'un joc i així abaratir costos. Tot això passava sobretot a la dècada dels setanta i principis dels vuitanta. Però finalment, amb l'arribada dels ordinadors personals i les consoles de sobretaula, es comença a desenvolupar un mercat per a jocs sobre plataformes genèriques.

A poc a poc van apareixent els motors gràfics - programes o mòduls encarregats del renderitzat d'un joc o d'un programa amb gràfics 2D o 3D - o fins i tot motors de joc - una plataforma per desenvolupar-hi un joc a sobre -. En un principi aquests motors s'utilitzaven dintre de la mateixa companyia que el creava. Per exemple LucasArts creà {SCUMM} ({Script Creation Utility for Manic Mansion}) \citep{WikiScumm} a l'hora que creava la seva aventura gràfica de "Point \& Click" Manic Mansion (1987). Aquest mateix programa fou utilitzat després en d'altres jocs com Indiana Jones i l'Última Creuada, LOOM, El Dia del Tentacle i tres jocs de la saga Monkey Island (fins al 1998).

Un pas més endavant el va dur Id Software amb el seu {\em id Tech}  \citep{WikiScumm}. Aquest motor - i les seves evolucions - no només es feu servir per fer jocs com Doom (1993) i Doom II (1994) d'Id Software, sinó que es va vendre a altres companyies per a fer altres jocs, tot i que aquests jocs serien molt semblants al Doom original, com serà el Half-Life (1998) basat en el Quake engine, una evolució de l'{\em id Tech}. Posteriorment fins i tot hi va haver companyies que basaven el seu negoci no en vendre jocs, sinó en vendre motors a altres companyies que els fessin servir; és doncs l'aparició definitiva dels motors com a Middleware.

Paral·lelament a l'evolució tecnològica ja esmentada, la metodologia de desenvolupament i els mateixos llenguatges de programació han anat evolucionant. L'evolució més important fou quan es va passar de programar bàsicament en {\bf C} a {\bf C++}. El canvi de paradigma, però, s'ha demostrat difícil i, tot i que l'ús de la metodologia orientada a objectes és predominant a quasi totes les àrees d'un motor, encara n'hi ha alguna on s'ha demostrat no ser la millor solució. A la figura \ref{fig:EsquemaEngine} es mostren algunes parts d'un {\em Engine} i com el mòdul de lògica usa els altres. D'aquesta manera un pot imaginar-se un {\em Engine} com a un proveïdor de serveis per al mòdul de lògica.


\begin{figure}
  \centering
  \includegraphics[width=0.58\linewidth]{./img/EsquemaEngine.png}
  \caption{Esquema genèric d'un Engine \label{fig:EsquemaEngine}}
\end{figure}

L'àmbit on més canvis hi ha hagut, en quan a metodologia de desenvolupament, i el que en aquest treball ens centrarem és en la definició de la lògica d'un joc. Les lògiques dels jocs, també conegudes com a mecàniques, són les normes que defineixen els jocs. Aquestes normes estan compostes per diverses peces: elements interactius i no interactius (com l'escenari del joc, els personatges, {\em attrezzo}, i altres), accions que poden fer els elements interactius i els objectius que té el jugador.

En aquest capítol s'explicarà primer l'evolució històrica en quan a programació de la lògica dels jocs. Seguidament agafarem la mostra que volem implementar i la definirem, i es mostraran alguns exemples de jocs i {\em Engines} que han implementat solucions semblants. Finalment s'esmentaràn els objectius d'aquest projecte.

\section{Evolució en les mecàniques dels jocs}

 En aquesta secció s'explicarà l'evolució que han anat prenent els mètodes que tenen els desenvolupadors per a definir aquestes normes; des de l'aproximació clàssica, jeràrquica i intuïtiva, agrupant els elements per característiques comunes, seguida d'un replantejament estructural, on se segueix una arquitectura menys intuïtiva, però més flexible, sense definir jerarquies de elements, amb característiques comunes, sinó definint cada característica per separat i després deixant que cada element agrupi les que necessita. I finalment presentant una arquitectura que trenca per complet l'aproximació clàssica sense deixar de ser una evolució del pas anterior.

\subsection{L'aproximació clàssica}

Quan un programador de {C++} o de qualsevol llenguatge orientat a objectes s'asseu davant el desafiament de programar la lògica d'un joc, llista les diferents entitats del joc (escenari, jugador, enemics, armes, etc.) i els distribueix en diferents grups i subgrups. Després en programa les funcionalitats comunes i acaba creant una jerarquia de classes que defineixen totes les entitats i les seves interrelacions. 

Aquesta aproximació sembla senzilla, però acaba comportant diversos problemes. Com s'explica a \citep[p.~719]{Gregory09}, les jerarquies de classes massa grans pateixen dels següents inconvenients:




\begin{itemize}
  \item {\bf Manteniment:} A més profunda és una classe dintre d'una jerarquia, més costa d'entendre, mantenir i modificar, ja que s'ha d'entendre tant ella com totes les seves classes superiors. Per exemple el fet de modificar una funció virtual aparentment innòcua, pot comportar trencar suposicions que faci alguna de les classes mare, cosa que du a errors subtils i difícils de detectar.
    
  \item {\bf Impossibilitat de descriure taxonomies multi-dimensionals:} Crear classes en forma d'arbre és molt pràctic i sobretot intuïtiu, especialment on a cada nivell es fan separacions respecte un criteri per cada nivell. Inicialment, ens plantegem una llista de grups a classificar: Vehicles contra Animals, Terrestres contra Marítims, etc. Aquí se'ns obra el primer dilema: Classifiquem primer entre animals i vehicles, o entre terrestres o marítims? Però el problema seriós arriba quan ens trobem amb classificacions que no havíem previst inicialment. Per exemple podríem havent classificat vehicles terrestres i aquàtics (figura \ref{fig:Vehicles}), i posteriorment podem haver d'afegir vehicles amfibis, repte que es pot intentar solucionar de les següents maneres:
  
  \begin{figure}
    \centering
    \includegraphics[width=0.58\linewidth]{./img/Vehicles.png}
    \caption{Vehicles classificats en terrestres i aquàtics \label{fig:Vehicles}}
  \end{figure}
  
    \begin{enumerate}
      \item {\bf Crear una nova classe hereva de vehicle \"{}VehicleAmfibi\"{}:} Solució òptima a primer cop d'ull, però comporta duplicar codi, ja que el codi necessari per fer que un vehicle vagi per terra ja es troba a {\em VehicleTerrestre}, per exemple. Això ens du a intentar la segona solució.
      
      \item {\bf Herència múltiple. El diamant de la mort:} La solució naïf del problema anterior seria crear la classe {\em VehicleAmfibi} hereva tant de {\em VehicleTerrestre} com de {\em VehicleAquàtic}, cosa que ens porta directament a l'herència múltiple. Com s'explica a \cite[p.~2]{Martin97}, l'herència múltiple causa diversos problemes, moltes vegades més grans que aquells que soluciona. Per exemple en aquest cas veuríem que la classe {\em VehicleAmfibi} heretaria dues vegades {\em Vehicle}, amb els problemes d'ambigüitat que això duria (figura \ref{fig:RombeMort}).
        
        
      \begin{figure}
        \centering
        \includegraphics[width=0.58\linewidth]{./img/RombeMort.png}
        \caption{Diamant de la mort \label{fig:RombeMort}}
      \end{figure}
        
      \item {\bf Classes Mix-in:} Una altra solució per crear taxonomies multi-dimensionals és crear un seguit de classes que aportin funcionalitat a diversos llocs de la jerarquia. Aquestes classes, per funcionar bé, cal que siguin heretades només per les fulles i que cap d'elles tingui una classe mare. Per tant diríem que una classe només pot tenir un "avi" en qualsevol cas. Aquesta solució comporta, sobretot, molta disciplina i acaba resultant en una solució molt semblant a "agregar" funcionalitat en comptes d'heretar-la (figura \ref{fig:MixIn}).
        
      \begin{figure}
        \centering
        \includegraphics[width=0.58\linewidth]{./img/MixIn.png}
        \caption{Esquema amb classes \"{}Mix-In\"{} \label{fig:MixIn}}
      \end{figure}
    \end{enumerate}
    
  \item {\bf Efecte bombolla:} A l'inici del disseny, les classes arrels - les més pròximes a l'inici de la jerarquia - són dissenyades inicialment amb poca funcionalitat. A mesura que avança el projecte, i davant del desig de compartir codi (i sobretot, no duplicar-lo), molta funcionalitat va pujant a la jerarquia fins que troba el \"{}comú denominador\"{}. A poc a poc, les classes arrels es van fent pesades fins que contenen la major part de funcionalitat, que les seves filles s'han d'encarregar d'activar correctament. Col·lateralment això fa que moltes classes acabin tenint una funcionalitat i unes variables que realment no necessiten, fent que el programa usi més memòria de la necessària, un problema especialment greu en jocs de consola, on la memòria és un bé molt escàs.
    
    
  
\end{itemize}

Per a més informació sobre els defectes de l'aproximació clàssica, consultar \cite{Wilson02}.

\subsection{De l'{\em és-un} al {\em conté-un}}

La primera millora, o petit canvi, que es proposa respecte l'aproximació anterior, és agregar la funcionalitat en comptes d'heretar-la. Si es vol crear un objecte amb moviment, que es renderitzi, que col·lisioni i que s'animi - un enemic, per exemple -, en l'aproximació clàssic la classe que representa aquest objecte heretaria d'una jerarquia on, més amunt o més avall, estigui implementada tota aquesta funcionalitat (figura \ref{fig:EnemicHerencia}). La millora proposada consta de fer que aquesta classe contingui aquelles que ens aportin cada una la funcionalitat que busquem (figura \ref{fig:EnemicAgregacio}).

\begin{figure}
  \centering
  \subfloat[Estructura per herència]{\label{fig:EnemicHerencia}\includegraphics[width=0.23\textwidth]{./img/EnemicHerencia.png}}
  \hspace{0.08\textwidth}
  \subfloat[Estructura per composició]{\label{fig:EnemicAgregacio}\includegraphics[width=0.27\textwidth]{./img/EnemicAgregacio.png}}
  \caption{Comparació d'herència i composició. \label{fig:HerenciaAgregacio}}
\end{figure}

Dintre d'aquest esquema, les classes que aporten funcionalitat són moltes vegades anomenades {\em Components} o {\em objectes-servidors}, ja que són les que componen els objectes finals.

Aquesta aproximació millora en certs aspectes l'anterior, com es mostra més endavant a la taula \ref{tab:comparacioFilosofies}, però encara té alguns inconvenients. Es tendeix a tenir una classe {\em GameObject} amb un punter a cada tipus de component, al qual se li afegeixen o treuen els components segons convingui - un {\em Enemic} contindrà el component {\em ObjecteAnimat} mentre que una {\em Habitació} no -; cosa que comporta primer el problema de qui s'ha d'encarregar de destruir els components - evident fins que arribes al punt que vols substituir un component en temps d'execució - i després la pèrdua d'eficiència que comporta haver de comprovar constantment si una instància conté o no cada component per fer-ne operacions.

Un pas més enllà es troba l'aproximació on cada tipus de {\em Component} implementa una interfície comuna molt bàsica i després el {\em GameObject} simplement conté un array dinàmic que els conté tots, i crida les seves funcions ordenadament. Aquesta aproximació comporta sovint el problema de que els components cal ordenar-los, ja que moltes vegades la seva funcionalitat no és commutativa.

Un últim pas en aquesta línia és eliminar per complet el {\em GameObject} i tenir arrays de components i lligar-los simplement per l'índex; Arrays o una implementació menys costosa en memòria com Taules Hash. Aquesta aproximació sovint anomenada com a {\em Model de Components Pur} \citep{Martin07} guanya sobretot en localitat de memòria. Normalment tots els components d'un tipus s'actualitzen alhora, en estar tots seguits en un array, el programa troba totes les dades juntes i s'executa més ràpidament que si les dades estiguessin dispersades.

Finalment, un gran avantatge d'un sistema d'aquest tipus és, com s'explica a \citep{Leonard99}, la possibilitat de, un cop definits els tipus de components i les seves propietats, crear un \"{}editor del joc\"{} que permeti, amb l'ajuda d'una interfície gràfica, crear, manipular, modificar, afegir, treure, etc. objectes i components del món de forma molt senzilla. Una eina d'aquestes característiques disminueix la dependència que hi ha entre dissenyadors i programadors i permet que cada un d'ells pugui fer la seva feina en paral·lel de la forma més eficient possible.

\subsection{De centrar-nos en els objectes, a centrar-nos en les propietats}

L'últim pas, que abstreu totes les idees anteriorment exposades, és trencar totalment l'esquema orientat a objectes. Aquest canvi s'inspira en part en l'arquitectura Model-Vista-Controlador. Concretament n'agafa el concepte de que cal separar les dades de la lògica que les controla. Si en féssim una base de dades hauríem de primer llistar tots els tipus de propietats que una entitat del joc pot tenir; de cada element d'aquesta llista l'anomenarem {\bf Component}, ja que les esmentades entitats seran formades a partir d'aquests. D'aquesta manera podem crear una taula per cada tipus de component que una entitat del joc pot tenir, i fer-ne una columna per cada propietat i una fila per cada objecte que el contingui, tenint cada objecte del joc, el que abans coneixíem com un {\em GameObject}, un identificador únic.

Aquesta aproximació no està pas mancada de defectes, i cal tenir-los en compte. El primer d'ells és la dificultat que hi ha d'establir relacions entre les {\em entitats} del joc, ja que no existeix una implementació en sí, d'aquesta entitat. Relacionat amb això hi ha el problema d'inicialitzar aquestes {\em entitats}, que s'ha de fer d'una forma més o menys aliena al mateix sistema. Finalment, el dilema més gran, és decidir on es programa el comportament de les entitats. Hi ha bàsicament 2 solucions, amb les seves variants:

\begin{enumerate}
  \item {\bf Dintre els mateixos components:} Cada propietat porta incorporada la seva funcionalitat d'alguna manera. Sigui directament dintre del mateix component, via mètodes en aquest - cosa que retorna a la metodologia orientada a objectes - o via els anomenats {\em sistemes}. Aquests s'encarreguen de recollir aquelles {\em entitats} que reuneixen unes condicions - normalment, tenir un o més components - i actualitzar-los.
    
  \item {\bf Via components especials:} En aquesta variant normalment es defineix un tipus especials que simplement conté un apuntador - o referència equivalent - a un script que implementi una interfície coneguda i que defineixi el comportament d'aquesta entitat.
    
\end{enumerate}

Finalment podem veure a la taula \ref{tab:comparacioFilosofies} una comparació de les característiques de cada arquitectura.

\begin{table}
  \begin{tabular}{ | p{0.15\textwidth} | p{0.20\textwidth} | p{0.20\textwidth} | p{0.40\textwidth} | }
    \hline
     &
     {\bf Manteniment} &
     {\bf Impossibilitat de descriure taxonomies multi-dimensionals} &
     {\bf Dependències externes} \\
     \hline
     
     {\bf Aproximació clàssica} &
     A més complexitat del sistema, més difícil és de modificar o ampliar. &
     Un cop s'han decidit quins atributs seran diferenciadors en cada punt de la jerarquia, canviar-los porta problemes. &
     Si volem modificar una dependència (per exemple, canviar la llibreria de física) caldrà modificar aquelles classes que encapsulin la física i adaptar totes aquelles hereves i en conseqüència cal fer molts canvis al programa, especialment si hi ha entitats que no fan servir aquesta característica però la duen incorporada, degut a l'efecte bombolla.\\
     \hline
     
     {\bf Composició de Propietats} &
     La complexitat vé relacionada amb la quantitat d'interrelacions entre les propietats, no amb el nombre total de propietats. &
     Una entitat pot ser reclassificada en qualsevol moment, i l'ampliació de classificacions no afecta a les ja existents. &
     A l'hora de modificar una llibreria externa només cal tocar els components que la necessitin i, si hi ha un canvi prou gran, les entitats que els continguin. \\
     \hline
     
     {\bf Separar les dades de la Lògica} &
     El sistema té una arquitectura més complexa, però al minimitzar les dependències és la més flexible. &
     Exactament igual que en el cas anterior. &
     Al dissenyar les dades del joc de forma independent de la seva implementació, aquesta es pot canviar sense més conseqüències. \\
     \hline
  \end{tabular}
  \caption{Comparació entre diferents arquitectures. \label{tab:comparacioFilosofies}}
\end{table}

\section{Sistema d'Entitats}

En aquesta secció donarem una definició, la nostra, del que és un sistema d'entitats. Tot seguit, es mostraran diferents implementacions que s'han fet, tant de Videojocs com d'{\em Engines}.

\subsection{Definicions}

Un sistema d'entitats està compost de diferents elements que cal entendre.

En primer lloc, les {\bf entitats}. Una {\bf entitat} en un videojoc és cada element que d'alguna manera apareix en el videojoc de forma concreta. Per exemple, l'escenari del videojoc, l'{\em avatar} del jugador, els enemics, les armes, els objectes que el jugador recull. Si féssim un llistat exhaustiu del joc dels escacs, donaríem les peces, el taulell i, si s'escau, el rellotge que marca el temps límit.

Seguidament, els {\bf components} són les característiques que defineixen les {\em entitats}. La vida d'un personatge, el model 3D de l'escenari i la velocitat d'un cotxe en són exemples, però també n'hi ha de més abstractes, com un component que defineix que un objecte fa so, o un altre que el marca com a {\em vehicle}. En resum, totes les característiques que les {\bf entitats} poden tenir s'agrupen en {\bf components}.

Finalment definim un {\bf sistema d'entitats} aquell {\bf sistema} que regeix el comportament d'un seguit d'{\bf entitats}, definides a partir d'uns {\bf components}. Cada entitat té associat com a màxim un component de cada tipus. El comportament d'aquestes entitats ve únicament regit per quins {\bf components} té associats i les dades que aquests contenen.

\subsection{Implementacions}

En aquesta secció farem un repàs de diverses implementacions de sistemes d'entitats que s'han fet anteriorment, analitzarem les seves característiques i els avantatges i inconvenients d'alguns.

Començarem amb un anàlisi d'alguns jocs que han seguit aquesta aproximació, seguidament parlarem de motors de jocs i finalment parlarem del nostre motor, i quines característiques tindrà que el diferenciïn dels anteriors.

\subsection{Jocs}

És difícil de rastrejar quins jocs tenen una estructura d'agregació en comptes d'herència, ja que moltes vegades no es fa pública la implementació d'aquestos. Però gràcies als anomenats {\em postmortemps} (Documents que s'escriuen un cop finalitzat un joc, on se n'analitza el desenvolupament, es busquen quines coses han anat bé i quines no i s'intenta fer auto-crítica de cara a millorar el desenvolupament dels següents projectes) podem tenir accés a la informació d'alguns jocs. A continuació una breu mostra d'alguns d'aquests, amb la referència d'on se'n poden trobar més detalls.

\begin{description}
  \item[Thief] \citep{Leonard99} Potser un dels exemples més antics, encara que segurament no el més antic, ja que el sistema és sempre una evolució d'algun motor usat anteriorment. Basa la seva força en facilitat d'edició mitjançant GUIs i una alta versatilitat, desenvolupant 2 jocs en paral·lel bona part del temps ({\em Thief} i {\em System Shock 2}). Es destaca que en aquests jocs no hi ha cap jerarquia d'objectes en el codi, sinó que tota es crea a partir de dades.
    
  \item[Tony Hawk's Pro Skater 3] \citep{West07} En aquest exemple veiem la necessitat d'una companyia de treure jocs amb una alta freqüència (un a l'any) i com, a poc a poc evolucionant cap a un sistema d'entitats fetes amb agregació de components. El canvi de jerarquies en el codi a objectes creats via dades fou lent i al principi contraproduent, però 2 anys després de començar el canvi els resultats es feren evidents, on el joc va guanyar en velocitat d'execució i els creadors en temps de producció.
    
  \item[Dungeon Siege] \citep{Bilas02} Potser dels més coneguts, sinó el més referenciat, Dungeon Siege anava equipat amb un sistema de components que permetia crear aquests components tant en {\em C++} com amb {\em Skirt}, un llenguatge propi del joc. Aquesta flexibilitat, i una aproximació a les dades molt similar a una Base de Dades Relacional va cridar molt la atenció a la GDC de San José (Un congrés de creadors de videojocs), potser quan els sistemes d'entitats van començar a agafar nom a la indústria.
    
\end{description}


\subsection{Motors de jocs}

Diversos motors de jocs han seguit aquesta filosofia, tant comercials, oberts o híbrids. En repassarem alguns.

\begin{description}
  \item[Unity] (\url{http://unity3d.com/}) Probablement el motor més ambiciós disponible sota aquesta filosofia. Seguint un model de negoci híbrid, ja que pots fer servir una versió limitada de forma gratuïta i accedir a diverses millores pagant, Unity aporta sobretot un editor molt potent, que permet fer gairebé qualsevol cosa que necessites des del mateix editor i veure com queda al joc quasi en temps real. Aquesta aproximació redueix moltíssim els temps de producció i facilita la feina sobretot als dissenyadors, per la facilitat que dóna a prototipar dissenys.
    
  \item[Frostbite Engine] (\url{http://en.wikipedia.org/wiki/Frostbite_Engine}) Motor comercial creat per EA DICE, sucursal sueca d'Electronic Arts, i usat principalment a la saga Battlefield des del 2008.
    
  \item[DtEntity, Ember, Grease] \citep{EntityWiki} Projectes OpenSource fets en C++, Flash i Python, respectivament. El seu desenvolupament encara no es pot considerar acabat, però contenen força característiques similars, com el desig de ser multi-plataforma i tenir un editor.
    
  \item[TyphonRT] (\url{http://www.typhon4android.org/}) En principi un Sistema d'Entitats similar al que aquí desenvoluparem, però a data de finalitzar aquest projecte encara no s'ha fet públic i s'espera el llançament per al tercer trimestre del 2011. Entre les seves característiques hi ha la programació via Java, compatibilitat amb Android 1.5+ i render via OpenGL/ES.
  
\end{description}

\section{Objectius del projecte}

L'objectiu del projecte és crear una plataforma per al desenvolupament de videojocs basada en els sistemes d'entitats. Aquesta plataforma ha de permetre la creació de diferents elements, com Components, Sistemes, Prototips, etc., que seran explicats en detall més endavant.

S'ha decidit que la millor manera de definir aquests elements és la de crear un llenguatge de programació que ho permeti. Aquest llenguatge s'anomenarà Quadriga, en honor a un dels esports amb més renom disputats als Jocs Olímpics clàssics.

A més, s'han marcat un seguit de característiques que es desitja que tingui la plataforma:

\begin{itemize}
  \item {\bf Open Source} \hfill \\
    S'ha decidit que aquest projecte ha de poder servir de base a qualsevol altre i ser el màxim d'obert possible. L'objectiu final és, doncs, fer una eina que qualsevol pugui fer servir.
    
  \item {\bf Independent} \hfill \\
    De tots els mòduls dels quals un {\em Engine} està format, Quadriga només n'implementarà un, el mòdul de lògica, i serà totalment independent dels altres.
    
  \item {\bf Multi-plataforma de forma nativa} \hfill \\
    Intentar que un programa desenvolupat usant aquest projecte sobre una plataforma, per exemple Windows, funcioni a qualsevol altre, per exemple GNU/Linux sense fer grans ajustaments.
    
  \item {\bf Permetre un desenvolupament ràpid un cop estiguin fets els components bàsics} \hfill \\
    L'objectiu és que sigui fàcil prototipar i iterar. El projecte ha de permetre crear peces de forma molt complerta i després usar-les de forma molt senzilla.
    
  \item {\bf Fàcilment paral·lelitzable} \hfill \\
    Veient la evolució actual dels processadors, sembla evident que els futurs programes per aprofitar les capacitats d'aquests han de poder executar tasques en paral·lel. Un dels objectius és minimitzar l'esforç que l'usuari dediqui a aquesta tasca donant el màxim d'elements ja fets.
\end{itemize}


En el capítol \ref{chap:Desenvolupament} es tractarà del disseny del llenguatge Quadriga, que ens permetrà definir {\em Components} i {\em Sistemes} així com executar un programa usant-los. També es parlarà de com és l'execució d'un programa així i com està construït tant el compilador com l'intèrpret. En el capítol \ref{chap:Resultats} es veurà l'exemple d'un joc creat a partir de Quadriga i s'analitzarà el rendiment en diversos sentits. Finalment, en el capítol \ref{chap:Conclusions} es revisarà el funcionament respecte del disseny inicial i es proposaran millores que s'hi podrien afegir; i els annexos fins i tot hi haurà informació sobre com crear un programa sobre Quadriga, d'on obtenir-ne el codi i fins i tot com col·laborar en la seva millora.